%% PLEASE DON'T CHANGE BELOW
\documentclass[10pt, twoside]{article}
\usepackage[paperwidth=17cm,paperheight=24cm,textheight=18cm]{geometry}
\pagestyle{empty} % Header and footer set by master document
\date{}           % no date for article, only for proceedings
%% PLEASE DON'T CHANGE ABOVE

% START EDITING HERE

% your title
\title{\LARGE \bfseries Visual Data Mining Using New RapidMiner Extensions}

% author(s)
\author{Jan Fabi\'{a}n \and Jan Motl
%\\Faculty of Information Technology\\
%Czech Technical University in Prague
}

\begin{document}
\maketitle\thispagestyle{empty}

\begin{abstract}
TBD.
Also fix the ``institute'' problem in author field.
\end{abstract}

\section{Introduction}

Information visualization is an effective and natural way to express data
and to make complex systems easier to understand.
This holds fully for data sets as well as for their models when talking about data mining.

RapidMiner has come with a decent possibilities for visualizing
multidimensional data for example, however, it has lacked
a universal straightforward tool for model visualization,
if not considering workarounds with creating data set with predicted labels.
[TODO: MENTION SOM]

In this paper, we present new possibilities of visual data mining in RapidMiner,
provided by two extensions -- the Self-Organizing Map and the Model Visualization Extension.

\subsection{Paper Structure}

...

\section{Self Organizing Map}

...

\section{Model Visualization}

\subsection{Motivation}

We might find interpretation of some types of models quite easy (e.g. decision trees),
however, there are other ones, for which these efforts are usually much less successfull
(e.g. neural nets, model ensembles).
Nevertheless, comprehensible model behaviour is a quite typical request on the output of the data mining process.

Structural visualization might help sometimes (once again, let's mention the decision trees),
however, in other cases (neural nets), it won't bring us any closer to our objective.
For the latter types of models, which we can consider ``black-box'' ones,
visualizations based on sensitivity analysis can provide a useful insight to the model behaviour.

The aim of the presented plugin was to fill in the existing gap
and to provide an easy way of inspecting any type of learned prediction model,
independently on its type.
In addition, with a focus on automation, an algorithm searching for
what is beleived to be areas of user's interest (in terms of the actual visualization setting) is provided.

\subsection{Visualizing Model Behaviour Using Sensitivity Analysis}

\section{Automated Visualization Optimization}

\subsection{Motivation}

\subsection{Estimating Model Reliability}

\subsection{Measuring Visualization Quality}

\subsection{The Genetic Algorithm}

\section{Described Features in RapidMiner Extensions}

\section{Future Work}

\section{Conclusion}

\end{document}
